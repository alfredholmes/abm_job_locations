\documentclass[a4paper,10pt]{article}
\usepackage[utf8]{inputenc}
\usepackage{hyperref}
\usepackage{graphicx}

%opening
\title{Building a Granular Dataset of UK Companies}
\author{Alfred Holmes}
\date{September 2018}
\begin{document}

\maketitle

\section{Predcting ONS Data from Companies House}
\subsection{Local Units}
Using local unit data one can infer many properties at the local authority and national levels. For example, if the size distribution of local units is known then employment levels can be calculated, and using this granular data one can test possible growth models for companies\footnote{to be explained in later section}.
\begin{figure}[!ht]
 \caption{Sic code ratio time series and histogram}
 \label{sic_ratio}
\end{figure}

Since, as shown in figure \ref{sic_ratio}, for each 2 digit SIC code\footnote{This is not true for 4 digit SIC} the ratio of companies house registered enterprises to local units is approximately constant. This suggests that is it reasonable to use these ratios to infer the number of local units from the companies house data.

\begin{figure}[!ht]
 \caption{Local unit predictions}
 \label{local_unit_predictions}
\end{figure}

Doing this creates a far more detailed data set of local units than is given by the ONS data 




\end{document}
